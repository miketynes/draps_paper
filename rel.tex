\section{Related Work}
Virtualization serves as one of the fundamental technologies in cloud computing systems. 
%In a virtualized computing environment, containers play an important role.
As a popular application, virtual machines (VMs) have been studied for decades.
However, in the reality, VMs suffer from noticeable
performance overhead, large storage requirement, and limited scalability~\cite{xu2014managing}.
More recently, containerization, a lightweight virtualization technique,  
is drawing increasing popularity from different aspects and on different 
platfroms~\cite{secpod, men2012interface, 
cheng2014efficiently, cheng2013qbdj, cheng2016efficient, cheng2014efficient, edos, bhimani2016, bhimani2017,tang2012gpu, du2015gpu, zhao2012mesh}.

The benefits and challenges of containerized systems have been studied in many aspects.
A comprehensive performance study is presented in ~\cite{felter2015updated}, where it explores the traditional
virtual machine deployments, and contrast them with the use
of Linux containers. The evaluation focuses on overheads and experiments 
that show containers' resulting performance to be equal or superior to VMs performances.
Although containers outperform VMs, 
the research~\cite{slacker} shows that the startup latency is considerably larger than expected.
This is due to a layered and distributed image architecture, in which copying package data accounts
for most of container startup time. The authors propose Slacker which can significantly reduce the startup latency.
While Slacker reduces the amount of copying and transferring packages, if the image is locally available, the
startup could be even faster. CoMICon~\cite{nathan2017comicon} addresses the problem by sharing the image in a cooperative manner. From different aspect, SCoPe~\cite{scope} tries to manage the provisioning time for large scale containers. 
It presents a statistical model, used to guide provisioning strategy, to characterize the provisioning time in terms of system features.

Besides the investigations on standalone containers, the cluster of containers is another important aspect in this field.
Docker Swarmkit~\cite{swarmkit} and Google Kubernetes~\cite{bernstein2014containers} are dominant cluster management tools in the market.
The authors of ~\cite{gog2016firmament}, first, conduct a comparison study of scalabilities under both of them. Then, firmament is proposed \added{[it will need to be clarified that firmament is the name of a technology ]}
to achieve low latency in large-scale clusters by using multiple min-cost max-flow algorithms. 
On the other hand, focusing on workload scheduling, the paper~\cite{kaewkasi2017improvement} describes an Ant Colony Optimization algorithm
for a cluster of Docker containers. However, the algorithm does not distinguish various containers, which usually have a divese requirements. \added{[more on how this is a disadvantage??]}

In this paper, we investigate the container orchestration in the prospective of resource awareness. 
While users can set limits on resources \added{[how?]}, containers are still competing for resources in a physical machine.
Starting from different images, the containers target various services, which results in different requirements on resources.
Through analyzing the dynamic resource demands, our work studies a node placement scheme that balance the resource usages in a 
heterogeneous cluster. 

%From the network security aspect, 
%authors in~\cite{wang2015benefit} build a virtual environment based on containers. The research shows that
%although the container performs well in terms of performance, it is in securable in
%terms of security control.


%Containerized applications have been studied in many research projects.
%becoming increasingly popular in both industry and academic.
%Containerization
%Due to the benefits in development and deployment brought by containers, lots of 
%big players stepped into this area, such as Amazon, Microsoft and Google, and 
%In~\cite{chung2016using}, the authors evaluate the performance
%of containers and virtual machines by running benchmarks. 