\section{Problem Formulation}
The \sol~ scheduler aims to optimize the container placement such that the available
resources on each worker node are maximized. In this paper, we assume that a container requires multiple resources such as memory, CPU, bandwidth, and I/O for running its services. Since the services and their workloads in a container change over time, the resource requirements in a container also exhibit temporal
dynamics. Therefore, we formulate the resource requirements of a container as a function of time.
Denote $r_i^{k}(t)$ as the $k$th resource requirement of the $i$th container at time $t$. Let $x_{i,j}=\{0, 1\}$ be the container placement indicator.
If $x_{i,j}=1$, the $i$th container is placed in the $j$th work node. Denote $W_j^k$ as the total amount of the $k$th
resource in the $j$th work node. Let $\mathcal{C}$, $\mathcal{N}$, $\mathcal{K}$ be the set of containers, work nodes,
and the resources, respectively. The utilization ratio of the $k$ resource in the $j$th work node can be expressed as
\begin{equation}\label{eq:work_util_ratio}
  u^k_j(t)=\frac{\sum_{i\in\mathcal{C}}x_{i,j}r_i^{k}(t)}{W_j^k}
\end{equation}
We assume that the utilization ratio of the $j$th work node is defined by its highest utilized resource. \added{[Is this DRF?]}
Then, the utilization ratio of the $j$th work node is $\max_{k\in\mathcal{K}}u^k_j(t)$.
The highest resource utilization among all the work nodes can be identified as
\begin{equation}\label{eq:util_worker}
  \nu=\max_{j\in\mathcal{N}}\max_{k\in\mathcal{K}}u^k_j(t).
\end{equation}
Since our objective when designing the \sol~ scheduler is to maximize the available
resources in each worker node, the \sol~ scheduling problem can be formulated as
\begin{align}\label{eq:sched_problem}
  \max_{x_{i,j}} &\;\;\;\;\; \nu \\
  s.t. &\;\;\;\;\; \sum_j x_{i,j}=1;\forall i\in\mathcal{C};\\
       &\;\;\;\;\; u^k_j(t)\leq 1, \forall k\in\mathcal{K}, \forall j\in\mathcal{N}.
\end{align}
The constraint in E.q. (4) requires that each container should be placed in
one worker node. The constrain in E.q. (5) enforces that the utilization ratio of any resource in a worker is less than one.
\added{[This feels lemma dropped in, should we introduce it?]}
\begin{lemma}
The \sol~ scheduling problem is NP-hard\deleted{ness}.
\end{lemma}
\begin{proof}
In proving the Lemma, we consider a simple case of the \sol~ scheduling problem in which the resource
requirements of each container are constant over time. The simplified \sol~ scheduling problem equals to
the multidimensional bin packing problem which is NP-hard~\cite{Bansal:2006:IAA,Meyerson:2013:OMLB,Im:2015:TBO}.
Hence, the lemma can be proved by reducing any instance of the multidimensional bin packing to the simplified \sol~
scheduling problem. For the sake of simplicity, we omit the detail proof in the paper. \added{[Should the full paper include the proof? It doesn't seem terribly difficult.]}
\end{proof}

\begin{comment}
\begin{table}[ht]
\caption{Notation Table}
 \centering
 \small
 \begin{tabular}{c l}
  \hline
  \hline
  $M_{id}$ / $W_{id}$ &  Manager ID / Worker ID \\
  \hline
  $S_{id}$ / $C_{id}$ & Service ID / Container ID \\
  \hline
  $W_j^{k}$ & The total $k$th resource on the $j$th worker node\\
  \hline
  $r_i^{k}(t)$ &  the $k$th resource requirement of \\
  & the $i$th container at time $t$\\
  \hline
  $\{KS\}$ & A set stores known services IDs \\
  \hline
  $DOM(S_{id})$ & A function records dominate resource attribute \\


  \hline
  \hline
\label{table:locations}
\end{tabular}
\end{table}

\end{comment}


\begin{comment}
\begin{table}[ht]
\caption{Notation Table}
 \centering
 \small
 \begin{tabular}{c l}
  \hline
  \hline
  $M_{id}$ / $W_{id}$ &  Manager ID / Worker ID \\
  \hline
  $S_{id}$ / $C_{id}$ & Service ID / Container ID \\
  \hline
  $W_{mem}$ & Current available memory on a specific worker \\
  \hline
  $W_{cpu}$ & Current available CPU on a specific worker \\
  \hline
  $W_{net}$ & Current available bandwidth on a specific worker \\
  \hline
  $W_{io}$ & Current I/O wait time on a specific worker \\
  \hline
  $\{KS\}$ & A set stores known services IDs \\
  \hline
  $DOM(S_{id})$ & A function records dominate resource attribute \\


  \hline
  \hline
\label{table:locations}
\end{tabular}
\end{table}

\end{comment} 