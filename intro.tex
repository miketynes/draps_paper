\section{Introduction}
\added{This sentence is too informal and awkward. }In the past few decades, we have witnessed a spectacular information explosion over the Internet.
Hundreds of thousands of users are consuming \added{[the internet isn't consumed...]} the Internet through various services, such as websites, mobile applications, and online games.
The service providers, at the back-end side, are supported by state-of-the-art infrastructures on the cloud, such as Amazon Web Service~\cite{aws} and
Microsoft Azure~\cite{azure}. 
Focusing on providing the services at scale, 
virtualization is one of the emerging \added{[is it really emerging? How old is it? If containerization is yesterday's news then isn't virtualization last year's news?]} technologies used in data centers and cloud environments to 
improve both hardware and development efficiency.

\replaced{Virtual machines are }{At the system level, the virtual machine is} a widely-adopted virtualization method~\cite{vm}, which isolate\deleted{s} CPU, memory, 
block I/O, network resources, etc~\cite{jithin2014virtual}.
In a large-scale system, however,
providing services through virtual machines\replaced{is largely intractable because it entails }{ would mean that the users are probably} 
running many duplicate instances of the same OS and \added{storing} many redundant boot volumes~\cite{medina2014survey}. 
Recent research shows that virtual machines suffer from noticeable performance overhead, large storage requirement, and 
limited scalability~\cite{xu2014managing}. 

To address these limitations, containers were designed for deploying
and running distributed applications without launching entire virtual machines. 
Instead, multiple isolated service units of the application, 
called containers, share the host operating system and physical resources.
The concept of container virtualization is yesterday's news;
Unix-like operating systems leveraged the technology for over a decade and modern 
big data processing \replaced{platforms}{plforms} utilize containers as a basic computing unit~\cite{wang2014fresh, wang2015omo, wang2017seina}. 
However, new containerization platforms, such as Docker, make it into the mainstream of application development.
Based on previously available open-source technologies 
(e.g. cgroup), Docker introduces a way of simplifying the tooling required to create and manage containers.
On a physical machine, containers are essentially just regular processes; in the system view, \added{[what does 'in the system view' mean here?]} that enjoy a virtualized resource environment, not only just CPU and
memory, but also bandwidth, ports, disk i/o, etc. 

We use ``Docker run image'' command to start a Docker container on physical machines. 
In addition to the disk image that we would like to
initiate, users can specify a few options, such as ``-m'' and ``-c'', to limit a container's access to resources. 
While options set a maximum amount, resource contention still happens among containers on every host machine.
%Starting with various images, a container can provide different services, which leads to a diverse resource demands. 
%For example, a clustering service, e.g. Kmeans, may need more computational power and a logging service, e.g. Logstash, 
%may request more bandwidth. 
Upon receiving ``Docker run'' commands from clients, the cluster, as the first step, should select a 
physical machine to host those containers. \added{[so on which machine is the user calling \texttt{docker run}? I am used to standalone docker] } The default container placement scheme, named Spread, uses a bin-pack strategy and tries to
assign a container on the node with the fewest running containers. 
While Spread aims to equally distribute tasks among all nodes, it omits two major characteristics of the system. First of all,
the nodes in a cluster do not necessary have to be identical with each other. It is a common setting to have multiple node types, 
in terms of total resource, in the cluster. For example, a cutting edge server can easily run more processes concurrently than a off-the-shelf desktop.
Secondly, the resource demands from containers are different. 
Starting with various images, services provided by containers are varied, which leads to a diverse resource demands. 
For instance, a clustering service, e.g. Kmeans, may need more computational power and a logging service, e.g. Logstash, 
may request more bandwidth. 

In this project, we propose a new container placement scheme, \sol, a Dynamic
and Resource-Aware Placement Scheme. Different from the default Spread scheme, \sol~ assigns containers based on current available 
resources in a heterogeneous cluster and dynamic demands from containers of various services. First, \sol~ identifies
the dominant resource type of a service by monitoring containers that offer this service. It, then, places the containers with 
complementary needs to the same machine in order to reduce the balance resource usages on the nodes. 
If one type of resource, finally, becomes a bottleneck in the system, it migrates the resource-intensive containers to other nodes.
Our main contributions are as follows:
%\vspace{-0.2in}
\begin{itemize}
 \item We introduce the concept of dominant resource type that considers the dynamic demands from different services. \added{[Are you really introducing it here? Hasnt this been done?]}
 \item We propose a complete container placement scheme, \sol, which assigns the tasks to appropriate nodes and balance resource usages
 in a heterogeneous cluster.
 \item We implement \sol~ into the popular container orchestration tool, Swarmkit, and conduct the experiment with 18 services in 4 types. \added{[clarify what a type is]}
 The evaluation demonstrates that \sol~ outperforms the default Spread and reduces usage as much as 42.6\% on one specific node. \added{[how cherry picked is this?]}
 
\end{itemize}

